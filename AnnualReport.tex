\documentclass{beamer}
\usepackage{graphicx}
\usepackage{textpos} 
\usepackage{calc}
%
%packages for presentation content
%
\usepackage{color}
\usepackage{epsfig}
\usepackage{amssymb,amsmath}
\usepackage{epstopdf}
\usepackage{etoolbox}
\usepackage{tikz}
%\usepackage{caption}
%\usepackage{subcaption}
\usepackage{xfrac}
\usepackage{ragged2e}
\usepackage{import}
\usepackage[doi=false,backend=bibtex]{biblatex}
%\usepackage[doi=false,backend=biber]{biblatex}
\bibliography{jogrady_bibdesk}
%\usepackage{biblatex}
%\addbibresource{jogrady_bibdesk.bib}
\DeclareGraphicsRule{.tif}{png}{.png}{`convert #1 `dirname #1`/`basename #1 .tif`.png}
%
%other commands for presentation content
%

\graphicspath{ {../} }
\graphicspath{ {./images/} }
\newcommand{\diagrampath}{./diagrams/}
\newcommand{\plotpath}{./plots}

\newcommand{\mathbi}[1]{\mathit{\mathbf{#1}}}
\newcommand\vstate[3]{%
	\mathbf{\underline{#1}}%
	\ifstrempty{#2}{}{[#2]}%
	\ifstrempty{#3}{}{\langle #3 \rangle}%
	}
\newcommand\sstate[3]{%
	\mathit{\underline{#1}}%
	\ifstrempty{#2}{}{[#2]}%
	\ifstrempty{#3}{}{\langle #3 \rangle}%
	}
%
%This outer block can be replaced by \usetheme{UTSA} if you have it
%
%\mode<presentation>
\useinnertheme[shadow=true]{rounded}
\useoutertheme{infolines}
%
%The following inner block can be replaced by \usecolortheme{UTSA} if you have it
%
%\mode<presentation>
\definecolor{UTSAdblue}{RGB}{0, 34, 68}
\definecolor{UTSAorange}{RGB}{244, 115, 33}

\definecolor{UTSAlblue}{RGB}{164, 179, 201}
\definecolor{UTSAyellow}{RGB}{243, 236, 122}
\definecolor{UTSAgray}{RGB}{169, 178, 177}
\definecolor{UTSAwgray}{RGB}{213, 210, 202}

\setbeamercolor{structure}{fg=UTSAdblue}

\setbeamercolor{palette primary}{fg=UTSAdblue,bg=UTSAyellow}
\setbeamercolor{palette secondary}{fg=UTSAdblue,bg=UTSAorange!80}
\setbeamercolor{palette tertiary}{fg=UTSAorange,bg=UTSAdblue!90}
\setbeamercolor{palette quaternary}{fg=UTSAdblue,bg=red}

\setbeamercolor{titlelike}{parent=palette quaternary}

\setbeamercolor{block title}{fg=UTSAdblue,bg=UTSAorange}
\setbeamercolor{block title alerted}{use=alerted text,fg=UTSAdblue,bg=alerted text.fg!75!bg}
\setbeamercolor{block title example}{use=example text,fg=UTSAdblue,bg=example text.fg!75!bg}

\setbeamercolor{block body}{parent=normal text,use=block title,bg=block title.bg!25!bg}
\setbeamercolor{block body alerted}{parent=normal text,use=block title alerted,bg=block title alerted.bg!25!bg}
\setbeamercolor{block body example}{parent=normal text,use=block title example,bg=block title example.bg!25!bg}

\setbeamercolor{sidebar}{bg=UTSAorange!70}
  
\setbeamercolor{palette sidebar primary}{fg=UTSAdblue}
\setbeamercolor{palette sidebar secondary}{fg=UTSAdblue!75}
\setbeamercolor{palette sidebar tertiary}{fg=UTSAdblue!75}
\setbeamercolor{palette sidebar quaternary}{fg=UTSAdblue}

\setbeamercolor*{separation line}{}
\setbeamercolor*{fine separation line}{}
%\mode
%<all>
%
% End of color theme block, copy and uncomment for a standalone color theme
%
\setbeamerfont{block title}{size={}}
\setbeamercolor{titlelike}{parent=structure,bg=UTSAorange}
%\mode
%<all>
%
%End of theme block, copy and uncomment for a theme file
%
%template mostly needs to define the UTSA logos
\def \logoLeft {\includegraphics[height=1cm,width=3cm,keepaspectratio]{UTSA_logo_left.pdf}}
\def \logoCenter{\includegraphics[height=1cm,width=4cm,keepaspectratio]{UTSA_logo_centered.pdf}}
%
%add logo to title boxes
%
\addtobeamertemplate{frametitle}{}{%
\begin{textblock*}{10mm}(\paperwidth-3cm,-1cm)
\logoLeft
\end{textblock*}}
%
%a bit of trickery to let us use a nice logo on the first page
%
\def \currentinst {\logoCenter}
%
\let\oldfootnotesize\footnotesize
\renewcommand*{\footnotesize}{\oldfootnotesize\tiny}
\setbeamertemplate{navigation symbols}{}%remove navigation symbols
%
%
\title{Peridynamic Beams, Plates, and Shells} %:\\ a non-ordinary state-based model}
\subtitle{a non-ordinary state-based model}
\author{James O'Grady}
\institute{\currentinst }
\date{Report, September 2014}

\listfiles

\begin{document}

{ %make title page with no footer (all the footer info is already there)
\setbeamertemplate{footline}{} 
\begin{frame}
  \titlepage
\end{frame}
}
\addtocounter{framenumber}{-1}
%
%switch to simple institution name for foot lines
%
\def \currentinst {UTSA}
%
% Content starts here
%
\section[Outline]{}
\begin{frame}
\frametitle{\;}
\tableofcontents
\end{frame}
%
%
%
\section{Motivation}
%
\subsection*{Material Failure}
%
\begin{frame}
  \frametitle{Material Failure Drives Design}
  
       \begin{columns}[t] % contents are top vertically aligned
     \begin{column}[B]{5cm} % alternative top-align that's better for graphics
  	Sometimes, \\
	we \textit{need} to model failure
	
	\vspace{.5cm}
	\centering
         \includegraphics[height = 4cm]{FEM_car_crash1.jpg}
          
          \vspace{0pt}
     \end{column}
       \begin{column}[B]{5cm} % each column can also be its own environment
       
       \centering
         \includegraphics[height=5.5cm]{platePenetrationBorvik}
          
         \vspace{0pt}
     \end{column}
     \end{columns}
     
            \begin{columns}[t] % contents are top vertically aligned
     \begin{column}[t]{5cm} % alternative top-align that's better for graphics

          \begin{center}
          \small{Car crash}
          \end{center}
          
          %\vspace{0pt}
     \end{column}
       \begin{column}[B]{5cm} % each column can also be its own environment

         \begin{center}
          \small{Ballistic plate penetration\footnotemark}
          \end{center}
          
         %\vspace{0pt}
  
     \end{column}
     \end{columns}
     
     \footnotetext{\fullcite{borvik1999ballistic}}
     
  \centering
  
  
\end{frame}
%
\section{Background}
%
\subsection{Classical Modeling Approaches}
%
\begin{frame}
  \frametitle{Classical Dynamics}
  \begin{center}
     PDE based methods like XFEM and RKPM solve the classical dynamics equation for continuum momentum conservation to find the relationship:\vspace{0.3cm}
    
    Displacement \(\leftrightarrow\) Strain \(\leftrightarrow\) Stress \(\leftrightarrow\) Force \(\leftrightarrow\) Accelleration
  \end{center}
  %\vspace{10 mm}
	
  Classical Dynamics has some inconvenient features
  \begin{itemize}
    \item Strain is the spatial derivative of displacement: \(\epsilon := \frac{\partial u}{\partial \mathbf{X}} \)
    \begin{itemize}
      \item not defined if displacements are discontinuous
    \end{itemize}
    \item Stress and Strain are \textit{local} functions of displacement: \(\sigma(\mathbf{x}) = f( \epsilon(\mathbf{x}))\)
    \begin{itemize}
      \item some materials exhibit nonlocal dependence
    \end{itemize}
  \end{itemize}
  
  So solving a PDE is begging the question: What is the spatial derivative at a crack?
  
\end{frame}
%
%
\subsection{Peridynamics}
%
\begin{frame}
  \frametitle{What is Peridynamics?}
  \begin{itemize}
    \item Greek: \textit{peri} - near or around, \textit{dyna} - force
    \item Force at a point is a function of the condition of the surrounding area (Strongly Nonlocal)
    \item Naturally handles discontinuous displacements
  \end{itemize}

  \centering
  \includegraphics[width=70mm]{pic_peri_body}
  
  \small{Two peridynamic bodies\footfullcite{foster2009state}}
  \vspace{.25cm}

\end{frame}
%
%
\begin{frame}
  \frametitle{Bond-based Models}
  Bond-based peridynamic Equation of Motion \footfullcite{silling2000reformulation} :
  \[ 
  \rho ( \mathbf{x} ) \mathbf{\ddot{u}} ( \mathbf{x} ) = \int_{\mathcal{H}} \mathbf{f}\left(\mathbf{u}\left(\mathbf{x'},\mathit{t}\right)-\mathbf{u}\left(\mathbf{x},\mathit{t}\right),\mathbf{x'}-\mathbf{x}\right)\mathit{dV}_ {\mathbf{x'}}+\mathbf{b}\left(\mathbf{x},\mathit{t}\right)
  \]
  The force at a point is an integral of contributions from all the relative displacements (\(\mathbf{u}\left(\mathbf{x'},\mathit{t}\right)-\mathbf{u}\left(\mathbf{x},\mathit{t}\right)\)) inside a horizon \(\mathcal{H}\). (no strain term \(\epsilon\))

  \centering
  \resizebox{.4\linewidth}{!}{\subinputfrom{\diagrampath}{BondForceCrop.eps_tex}}
 
  With a linear elastic bond force, the peridynamic model can reduce to a classical linear elastic solid with \(\nu=\frac{1}{4}\) or a 2D plate with  \(\nu=\frac{1}{3}\).
  \vspace{.1cm}
\end{frame}
%
%
\begin{frame}
  \frametitle{State-based Models}
In a state-based model, the force between two points can depend on the behavior of the other bonds surrounding them.
%
\begin{equation*}
\rho(\mathbf{x})\mathbf{\ddot{u}}(\mathbf{x},t) = \int_\mathcal{H} \left\{\vstate{T}{\mathbf{x},t}{\mathbf{q}-\mathbf{x}}-\vstate{T}{\mathbf{q},t}{\mathbf{x}-\mathbf{q}}\right\} \; dV_q + \mathbf{\hat{b}}(\mathbf{x},t),
\end{equation*}
%
where \(\vstate{T}{}{}\) is the vector state force function.\vspace{0.3cm}

\(\vstate{T}{}{}\) need not be oriented along the bond, this \textit{non-ordinary} state-based model could represent rotational springs between pairs of bonds.
%
\begin{figure}
\subinputfrom{\diagrampath}{bondPair.eps_tex}
\end{figure}
%
%
\end{frame}
%
%
\begin{frame}
  \frametitle{Model Types}
%
\centering
\includegraphics[width=8cm]{PDmodelTypes}

\small{Illustration of the three types of peridynamic models, from specific to general \footfullcite{silling2007peridynamic}}
%
\end{frame}
%
\subsection{Thin Features}
%
\begin{frame}
  \frametitle{Thin Features}
  Thin features present challenges for 3D solid models. 
  To accurately capture some material behavior, discretization must be very dense.
     \vspace{0.3cm}
  
     \begin{columns}[t] % contents are top vertically aligned
       \begin{column}[T]{5cm} % each column can also be its own environment
       \centering
       Finite Element thin features can be greatly simplified
       
       \includegraphics[height = 4cm]{ansysShell}
       
       \tiny{ANSYS brochure}
     
     \end{column}
     \begin{column}[T]{5cm} % alternative top-align that's better for graphics
       \centering
       Peridynamics has no equivalent models
       
       \includegraphics[height = 4cm,clip=true,trim= 15cm 1.5cm 0cm 0]{LittlewoodCylinder}
       \footnotemark
     \end{column}
     \end{columns}
     Thin models use the same failure modeling techniques as solid models
     
     \footnotetext{Image from \fullcite{littlewood2010simulation}}

\end{frame}
%
%
\section{Objectives}
%
\begin{frame}
  \frametitle{Objectives}
    Lay the foundation:
    \begin{itemize}
    \item Create peridynamic beam, plate, and shell models
    \item Verify correspondence to classical continuum models
  \end{itemize}
  Ultimate Goal:
  \begin{itemize}
    \item Simulate thin feature behavior
    \item Develop and validate failure models
    \item Simulate thin feature failure
  \end{itemize}
\end{frame}
%
%
\section{Methodology}
%
\subsection{Bond-pair Models}
%
\begin{frame}
  \frametitle{Bond Pair Model}
Starting with Silling's proposed non-ordinary State-based model\footfullcite{silling2007peridynamic}
%
\begin{figure}
\subinputfrom{\diagrampath}{bondPair.eps_tex}
\end{figure}
%
in which
%
\begin{equation*}
\vstate{T}{}{\boldsymbol{\xi}} =\frac{\alpha}{|\vstate{Y}{}{\boldsymbol{\xi}}|} \frac{\vstate{Y}{}{\boldsymbol{\xi}}}{|\vstate{Y}{}{\boldsymbol{\xi}}|} \times \left[\frac{\vstate{Y}{}{\boldsymbol{\xi}}}{|\vstate{Y}{}{\boldsymbol{\xi}}|} \times \frac{\vstate{Y}{}{-\boldsymbol{\xi}}}{|\vstate{Y}{}{-\boldsymbol{\xi}}|}\right]
\end{equation*}
%
This force state is the Fr\'{e}chet derivative of the energy state
%
\begin{equation*}
\sstate{w}{x}{\boldsymbol{\xi}} = \alpha [1 + \cos(\sstate{\theta}{x}{\boldsymbol{\xi}}) ]
\end{equation*}
%
\end{frame}
%
%
\begin{frame}
  \frametitle{Continuous Bond Pair Beam}
Start by modeling a 1D beam in bending.

The deformed bond-pair angle is:

\begin{equation*}
\label{eq:beamdtheta}
\theta(\vstate{Y}{x}{\xi},\vstate{Y}{x}{\mathbf{-\xi}}) \approx \pi-\frac{y(x+\xi)-2y(x)+y(x-\xi)}{\xi}
\end{equation*}

which is very similar to the finite difference second derivative

\begin{equation*}
\label{eq:P}
\theta(\vstate{Y}{}{\xi},\vstate{Y}{}{\mathbf{-\xi}}) \approx \pi-\xi \frac{\partial^2 y}{\partial x^2} =  \pi-\xi \kappa
\end{equation*}

The resulting total strain energy density

\begin{equation*}
W(x) \approx \int_{-\delta}^\delta \omega(\xi)\alpha\frac{\xi^2}{2}\kappa^2 d\xi = \frac{\alpha}{2}\kappa^2 \int_{-\delta}^\delta \omega(\xi)\xi^2 d\xi,
\end{equation*}

in which \(\omega(\xi)\) is a weighting function

%
\end{frame}
%
%
\begin{frame}
  \frametitle{Bond Pair Beam Energy}
%
Choosing \(\alpha\) carefully results in the classical Euler beam strain energy:
%
\begin{align*}
\alpha = \frac{EI}{m} ;\; m=\int_{-\delta}^\delta \omega(\xi)\xi^2 d\xi \implies W=\frac{EI}{2}\kappa^2
\end{align*}
%
or the discrete version: 
%
\begin{align*}
\label{eq:discretebeam}
\alpha &= \frac{EI\; \Delta x}{m} ;\; m=\sum_{i=1}^n \omega(\xi_i)\xi_i^2 \implies \nonumber \\
W&=\Delta x \sum_{i=1}^n \frac{EI}{2}\left(\frac{y(x+\xi_i)-2y(x)+y(x-\xi_i)}{\xi_i}\right)^2
\end{align*}
%
\end{frame}
%
%
\begin{frame}
  \frametitle{Bond Pair Damage Models}
  Determine critical angle $\theta_c$ from $\delta$, $\epsilon_c$, and $t$.
  
  Brittle material: bond pair ceases to exist
  
  Nonlinear elastic material: same moment regardless of angle
  \[ 
|\vstate{T}{}{\xi}| = 
  \begin{cases}
    \alpha \frac{\sin(\theta(\vstate{Y}{}{\xi},\vstate{Y}{}{\mathbf{-\xi}}))}{|\vstate{Y}{}{\xi}|} & \quad \text{if } \theta < \theta_c\\
    \alpha \frac{\sin(\theta_c)}{|\vstate{Y}{}{\xi}|} & \quad \text{if } \theta \geq \theta_c\
  \end{cases}
\]
 Elastic perfectly-plastic material: track the plastic deformation \(\theta^p (\xi) = \theta-\theta_c\) and let \(\theta^e (\xi) = \theta-\theta^p\)
 \[ 
|\vstate{T}{}{\xi}| = 
  \begin{cases}
    \alpha \frac{\sin(\theta^e(\vstate{Y}{}{\xi},\vstate{Y}{}{\mathbf{-\xi}}))}{|\vstate{Y}{}{\xi}|} & \quad \text{if } \theta^e < \theta_c\\
    \alpha \frac{\sin(\theta_c)}{|\vstate{Y}{}{\xi}|} & \quad \text{if } \theta^e \geq \theta_c\
  \end{cases}
\]
%
\end{frame}
%
\begin{frame}
  \frametitle{Beam Results}
     \begin{columns}[T] % contents are top vertically aligned
     \begin{column}[T]{5cm} % alternative top-align that's better for graphics
 	 \centering
	Elastic
	 \resizebox{\linewidth}{!}{\input{\plotpath/elastic_h20_g2000.pgf}}
     \end{column}
     \begin{column}[T]{5cm} % each column can also be its own environment
	\centering
	Elastic-Plastic
	\resizebox{\linewidth}{!}{\input{\plotpath/eppu_h10_g2000.pgf}}
     \end{column}
     \end{columns}
\end{frame}
%
%
\begin{frame}
     \begin{columns}[C] % contents are top vertically aligned
     \begin{column}[C]{3cm} % alternative top-align that's better for graphics
%       \centering
 	 \centering
	Brittle Beam:
     \end{column}
     \begin{column}[C]{6cm} % each column can also be its own environment
	\centering
	\resizebox{\linewidth}{!}{\input{\plotpath/brittle_h10_n200.pgf}}
     
     \end{column}
     \end{columns}
\end{frame}
%
%
\begin{frame}
  \frametitle{Bond Pair Plate}
    The same equations can model a peridynamic plate in bending:
     \begin{columns}[T] % contents are top vertically aligned
     \begin{column}[T]{6cm} % alternative top-align that's better for graphics
%       \centering
        \begin{figure}
        \vspace{-9mm}
        \resizebox{0.9\linewidth}{!}{\subinputfrom{\diagrampath}{continuousPlate.eps_tex}}
        \end{figure}
     \end{column}
       \begin{column}[T]{5cm} % each column can also be its own environment
%      The same equations can model a peridynamic plate in bending:
%       
       \vspace{5mm}
       By choosing \(\alpha\) as before, we find that this matches the classical strain energy for a plate with \(\nu = \sfrac{1}{3}\)
     
     \end{column}
     \end{columns}
%
%By choosing \(\alpha\) as before, we find that this matches the classical strain energy for a plate with \(\nu = \sfrac{1}{3}\)
%
\begin{align*}
\alpha &= \frac{c}{m} ;\; c= \frac{G t^3}{6 \pi} ;\; m=\int_{0}^\delta \omega(r)\frac{r^3}{2} dr ;\; \nu=\frac{1}{3} \implies \notag\\
W&=\frac{G t^3}{12(1-\nu)}\left(\left(\kappa_1\right)^2 +\left(\kappa_2\right)^2 +2\nu\left(\kappa_1\kappa_2\right) +2(1-\nu)\left(\kappa_3\right)^2 \right) 
\end{align*}
%
\end{frame}
%
%
\begin{frame}
  \frametitle{In-Plane Deformation}
  Pure bending model does not resist in-plane stretch or shear, but we can combine it with an extension-based model
  
  \vspace{10mm}  
%  \centering
  \begin{centering}
  
  \resizebox{0.6\linewidth}{!}{\subinputfrom{\diagrampath}{bondPairCombinedV.tex}}
  
  \end{centering}
  
  Failure of bond-pairs and failure of extension bonds can be coupled so that either mode of damage reduces both modes of stiffness

%
\end{frame}
%
%
\begin{frame}
  \frametitle{Arbitrary Poisson's Ratio}
  Bending can be divided into 2 types:
  
  \resizebox{\linewidth}{!}{\input{\plotpath/BendingDecomp.pgf}}
  
  A bending ``pressure'' proportional to the isotropic curvature \(\bar{\boldsymbol{\kappa}}\) allows simulation of arbitrary $\nu$
  
\begin{equation}
%    \bar{\boldsymbol{\kappa}} &= \frac{1}{m} \int_0^\delta \int_0^{2\pi}\omega(\xi)\frac{\vstate{Y}{}{\boldsymbol{\xi}}+\vstate{Y}{}{\boldsymbol{-\xi}}}{\xi^2} \xi {\rm d}\phi {\rm d}\xi ;
     \vstate{T'}{}{\boldsymbol{\xi}}=\frac{8G}{m}\frac{h^3}{12}\frac{3\nu-1}{1-\nu}\frac{\omega(\boldsymbol{\xi})}{\xi^2} \bar{\boldsymbol{\kappa}} \notag
\end{equation}
%
%
\end{frame}
%
%
\section{Preliminary Results}
%
%
\begin{frame}
  \frametitle{Plate Results}
     \begin{columns}[T] % contents are top vertically aligned
     \begin{column}[T]{5cm} % alternative top-align that's better for graphics
 	 \centering
	Tension-Stiffening
	\resizebox{\linewidth}{!}{\input{\plotpath/plateStiffening.pgf}}
     \end{column}
     \begin{column}[T]{5cm} % each column can also be its own environment
	\centering
	Arbitrary $\nu$
	\resizebox{\linewidth}{!}{\input{\plotpath/elasticPlatePoissonEffect.pgf}}
     \end{column}
     \end{columns}
\end{frame}
%
%
\begin{frame}
     \begin{columns}[C] % contents are top vertically aligned
     \begin{column}[C]{3cm} % alternative top-align that's better for graphics
%       \centering
 	 \centering
	Brittle Plate:
     \end{column}
     \begin{column}[C]{6cm} % each column can also be its own environment
	\centering
	\resizebox{\linewidth}{!}{\input{\plotpath/SingleTorsion.pgf}}
     
     \end{column}
     \end{columns}
\end{frame}
%
%

\section{Conclusion}
%
\subsection{Summary}
%
\begin{frame}
\frametitle{Accomplished}
  The first state-based peridynamic thin features
  \begin{itemize}
    \item Strain energy equivalence demonstrated
    \item Failure models proposed
    \item Numerical models coded and evaluated
    \item Good beam, plate, and flat shell results
  \end{itemize}

\end{frame}
%
\subsection{Ongoing Work}
%
\begin{frame}
\frametitle{In progress}
  Virtual point pairing
  \begin{itemize}
    \item Irregular discretization
    \item Curved beams, plates, shells
  \end{itemize}
  Energy rate based failure models

\end{frame}
%
\begin{frame}
  \frametitle{Questions?}
\end{frame}
%

\newcounter{finalframe}
\setcounter{finalframe}{\value{framenumber}}

\setcounter{framenumber}{\value{finalframe}}
%
\end{document}
