% Content starts here
%
%
\section{Motivation}
%
\subsection*{Material Failure}
%
\begin{frame}
  \frametitle{Material Failure}
  
     \begin{columns}[T] % contents are top vertically aligned

     \column{0.55\textwidth}
     %
     \vspace{-0.5cm}
     \begin{figure}
        \centering
        \includegraphics[width=0.55\textwidth]{southwest.jpg}
        \caption{Fuselage Rupture. Image c/o AP.}
     \end{figure}
	 %
     \vspace{-0.5cm}
     \begin{figure}
        \centering
        \includegraphics[width=0.55\textwidth]{FEM_car_crash1.jpg}
        \caption{Car crash}
     \end{figure}

     \column{0.45\textwidth}
       
     \begin{figure}
       \centering
       \includegraphics[width=0.7\textwidth]{plateperf2.jpg}
       \caption{Plate perforation}
     \end{figure}
     \end{columns}
     
     \footcite{borvik1999ballistic}
  
\end{frame}
%
\section{Background}
%
%
\subsection{Peridynamics}
%
\begin{frame}
  \frametitle{Peridynamics}
  \begin{center}
    \justify
    ``\ldots seeks to unify the mechanics of continuous media, particles, and cracks with a single mathematically consistent set of equations.''
  \end{center}

  \begin{figure}
      \centering
      \includegraphics[width=0.7\textwidth]{crack_branch.jpg}
      \caption{Quote \& image c/o S.~Silling, SNL}
  \end{figure}
  

\end{frame}
%
%
%
\begin{frame}
  \frametitle{Peridynamic EOM}
%
\begin{equation*}
\rho(\mathbf{x})\mathbf{\ddot{u}}(\mathbf{x},t) = \int_\mathcal{H} \left\{\vstate{T}{\mathbf{x},t}{\mathbf{x'}-\mathbf{x}}-\vstate{T}{\mathbf{x'},t}{\mathbf{x}-\mathbf{x'}}\right\} \; dV_{\mathbf{x'}} + \mathbf{b}(\mathbf{x},t),
\end{equation*}
%
\begin{figure}
    \centering
    \scalebox{0.6}{\input{./diagrams/peri_body.tex}}
\end{figure}
\footcite{silling:psa}
%
%
\end{frame}
%
%
\begin{frame}
  \frametitle{Constitutive Model Types}
%
\begin{figure}
    \centering
    \includegraphics[width=0.8\textwidth]{PDmodelTypes}
\end{figure}
\footcite{silling:psa}

%
\end{frame}
%
\subsection{Thin Features}
%
\begin{frame}
  \frametitle{Thin Features}
  
     \vfill
     \begin{columns}[b] % contents are top vertically aligned
       
       \column{0.5\textwidth}

       \begin{figure}
           \centering
           \includegraphics[width=0.75\textwidth]{ansysShell}
           \caption{FEA shells. Image c/o ANSYS manual.}
       \end{figure}
       
       \column{0.5\textwidth}

       \vspace{-0.8cm}
       \begin{figure}
           \centering
           \includegraphics[width=0.6\textwidth]{littlewood_cylinder2.jpg}
           \caption{No PD equivalent}
       \end{figure}
     
     \end{columns}
     
     \footcite{littlewood2010}

\end{frame}
%
%
\section{Objectives}
\subsection{}
%
\begin{frame}
\frametitle{Objectives}

   \begin{itemize}
        \item PD constitutive models for beam, plates, shells
        \item Simulate material failure in thin features
        \item Explore complexities of non-ordinary material models, meta-materials\ldots?
  \end{itemize}

\end{frame}
%
%
\section{Results}
%
\subsection{Bond-pair Models}
%
\begin{frame}
\frametitle{Bond Pair Model}
%
\begin{figure}
\subinputfrom{\diagrampath}{bondPair.eps_tex}
\end{figure}
%
in which
%
\begin{equation*}
\vstate{T}{}{\boldsymbol{\xi}} =\frac{\alpha}{|\vstate{Y}{}{\boldsymbol{\xi}}|} \frac{\vstate{Y}{}{\boldsymbol{\xi}}}{|\vstate{Y}{}{\boldsymbol{\xi}}|} \times \left[\frac{\vstate{Y}{}{\boldsymbol{\xi}}}{|\vstate{Y}{}{\boldsymbol{\xi}}|} \times \frac{\vstate{Y}{}{-\boldsymbol{\xi}}}{|\vstate{Y}{}{-\boldsymbol{\xi}}|}\right]
\end{equation*}
%
\footcite{silling:psa}
\footcite{jogrady2014a}
%
\end{frame}
%
%
\begin{frame}
  \frametitle{Bond Pair Beam Energy}
\vfill

\begin{equation*}
W(x) \approx \int_{-\delta}^\delta \omega(\xi)\alpha\frac{\xi^2}{2}\kappa^2 d\xi = \frac{\alpha}{2}\kappa^2 \int_{-\delta}^\delta \omega(\xi)\xi^2 d\xi,
\end{equation*}
%
\vspace{0.5cm}
\begin{center}
Choosing \(\alpha\) carefully:
\end{center}
%
\begin{align*}
\alpha = \frac{EI}{m} ;\; m=\int_{-\delta}^\delta \omega(\xi)\xi^2 d\xi \implies W=\frac{EI}{2}\kappa^2
\end{align*}
%

\footcite{jogrady2014a}
\end{frame}
%
%
\begin{frame}
  \frametitle{Plasticity and Damage}

  
  \begin{center}
      Brittle material
  \end{center}
  
\[ 
|\vstate{T}{}{\xi}| = 
  \begin{cases}
    \vstate{T}{}{\xi} & \quad \text{if} \quad \theta < \theta_c \\
    0 & \quad \text{if} \quad \theta \geq \theta_c\
  \end{cases}
\]

  \begin{center}
     Elastic perfectly-plastic material \\
  \end{center}

\[ 
|\vstate{T}{}{\xi}| = 
  \begin{cases}
    \alpha \frac{\sin(\theta^e(\vstate{Y}{}{\xi},\vstate{Y}{}{\mathbf{-\xi}}))}{|\vstate{Y}{}{\xi}|} & \quad \text{if} \quad \theta^e < \theta_c\\
    \alpha \frac{\sin(\theta_c)}{|\vstate{Y}{}{\xi}|} & \quad \text{if} \quad \theta^e \geq \theta_c\
  \end{cases}
\]
%
\footcite{jogrady2014a}
\end{frame}
%
\begin{frame}
  \frametitle{Beam Results}

     \begin{columns}[T] % contents are top vertically aligned

     \column{0.5\textwidth}

     \begin{figure}
         \centering
         \resizebox{\linewidth}{!}{\input{\plotpath/elastic_h20_g2000.pgf}}
         \caption{Elastic}
     \end{figure}

     \column{0.5\textwidth}

     \begin{figure}
        \centering
        \resizebox{\linewidth}{!}{\input{\plotpath/eppu_h10_g2000.pgf}}
        \caption{Elastic-Plastic}
     \end{figure}

     \end{columns}
\footcite{jogrady2014a}
\end{frame}
%
%
\begin{frame}

\frametitle{Beam Failure}
     \begin{figure}
        \centering
        \scalebox{0.4}{\input{\plotpath/brittle_h10_n200.pgf}}
     \end{figure}
\footcite{jogrady2014a}
\end{frame}
%
%
\begin{frame}
  \frametitle{Bond Pair Plate}

    \begin{center}
        Extend beam model to 2D
    \end{center}

     \begin{columns}[T] % contents are top vertically aligned

     \column{0.5\textwidth}
        \vspace{-1.5cm}
        \begin{figure}
            \centering
            \resizebox{0.9\linewidth}{!}{\subinputfrom{\diagrampath}{continuousPlate.eps_tex}}
        \end{figure}

     \column{0.5\textwidth}
       \vfill
       \begin{itemize}
           \item Choosing \(\alpha\) by matching strain energy with \(\nu = \sfrac{1}{3}\)
       \end{itemize}
     
     \end{columns}
%
%By choosing \(\alpha\) as before, we find that this matches the classical strain energy for a plate with \(\nu = \sfrac{1}{3}\)
%
\begin{align*}
\alpha &= \frac{c}{m} ;\; c= \frac{G t^3}{6 \pi} ;\; m=\int_{0}^\delta \omega(r)\frac{r^3}{2} dr ;\; \nu=\frac{1}{3} \implies \notag\\
W&=\frac{G t^3}{12(1-\nu)}\left(\left(\kappa_1\right)^2 +\left(\kappa_2\right)^2 +2\nu\left(\kappa_1\kappa_2\right) +2(1-\nu)\left(\kappa_3\right)^2 \right) 
\end{align*}

\footcite{jogrady2014b}
%
\end{frame}
%
%
\begin{frame}
  \frametitle{In-Plane Deformation}
  
  \vfill
  \begin{figure}
      \centering
      \resizebox{0.9\linewidth}{!}{\subinputfrom{\diagrampath}{bondPairCombinedV.tex}}
  \end{figure}
%
\end{frame}
%
%
\begin{frame}

  \frametitle{Arbitrary Poisson's Ratio}

  
  \resizebox{\linewidth}{!}{\input{\plotpath/BendingDecomp.pgf}}
  
  A bending ``pressure'' proportional to the isotropic curvature \(\bar{\boldsymbol{\kappa}}\) allows simulation of arbitrary $\nu$
  
\begin{equation}
%    \bar{\boldsymbol{\kappa}} &= \frac{1}{m} \int_0^\delta \int_0^{2\pi}\omega(\xi)\frac{\vstate{Y}{}{\boldsymbol{\xi}}+\vstate{Y}{}{\boldsymbol{-\xi}}}{\xi^2} \xi {\rm d}\phi {\rm d}\xi ;
     \vstate{T'}{}{\boldsymbol{\xi}}=\frac{8G}{m}\frac{h^3}{12}\frac{3\nu-1}{1-\nu}\frac{\omega(\boldsymbol{\xi})}{\xi^2} \bar{\boldsymbol{\kappa}} \notag
\end{equation}
%
\footcite{jogrady2014b}
\end{frame}
%
%
%
\begin{frame}
  \frametitle{Plate Results}
    \begin{columns}[T] % contents are top vertically aligned
     
    \column{0.5\textwidth} % alternative top-align that's better for graphics

    \begin{figure}
        \centering
        \resizebox{\linewidth}{!}{\input{\plotpath/plateStiffening.pgf}}
        \caption{Tension-Stiffening}
    \end{figure}

    \column{0.5\textwidth} % alternative top-align that's better for graphics

    \begin{figure}
        \centering
        \resizebox{\linewidth}{!}{\input{\plotpath/elasticPlatePoissonEffect.pgf}}
        \caption{Arbitrary $\nu$}
    \end{figure}
    
    \end{columns}
\footcite{jogrady2014b}
\end{frame}
%
%
\begin{frame}
\frametitle{Crack propagation}

\vspace{-0.42cm}
\begin{figure}
	\centering
	\scalebox{0.28}{\input{\plotpath/SingleTorsion.pgf}}
\end{figure}
     
\footcite{jogrady2014b}
\end{frame}
%
%
\section{Conclusions}
%
\subsection{Summary}
%
\begin{frame}
\frametitle{Summary}
  \begin{itemize}
    \item Plate and beam non-ordinary state-based models that recover classical response.
    \item Simple failure models demonstrated.
  \end{itemize}

\end{frame}
%
\subsection{Ongoing Work}
%
\begin{frame}
  \frametitle{In progress}
  \begin{itemize}
    \item Irregular discretizations
    \item Curved shells
    \item Complex failure models
    \item Meta-materials???
  \end{itemize}

\end{frame}
%
\subsection{Acknowledgment}
\begin{frame}
  \frametitle{Acknowledgment}
  \vfill
  \begin{figure}
      \includegraphics[width=0.3\textwidth]{james.jpg}
      \caption{James O'Grady}
  \end{figure}
\end{frame}
%

\subsection{Questions}
\begin{frame}
  \frametitle{}
  \vfill
  \begin{center}
      Questions?
  \end{center}
\end{frame}
%

\newcounter{finalframe}
\setcounter{finalframe}{\value{framenumber}}

\setcounter{framenumber}{\value{finalframe}}
%
